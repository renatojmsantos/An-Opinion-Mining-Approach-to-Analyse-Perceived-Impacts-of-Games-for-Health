
\par O primeiro resultado em análise é a deteção da presença do vocabulário \textit{WHOQOL-100} nos comentários apresentados. 

\par Na tabela \ref{tab:presenca-wholqol100} pode constatar os resultados que evidenciam que tanto com todos os participantes como dividindo por grupos de áreas profissionais (nos casos de duas pessoas ou mais), obtemos um \begin{math}\alpha\end{math} de \textit{Krippendorff} que permite concluir que existe uma elevada fiabilidade.

\begin{table}[H]
\renewcommand{\arraystretch}{1.1}
\centering
\setlength{\leftmargini}{0.05cm}
\begin{tabular}{|m{2.0cm}|m{2.0cm}|m{2.0cm}|m{2.0cm}|m{2.0cm}|m{2.0cm}|}
\hline
\textbf{}&\textbf{Todos}&\textbf{Medicina}&\textbf{Psicologia}& \textbf{Ciências Farmacêuticas} & \textbf{> 10 anos exp. profis.}\\ \hline

\begin{math}\alpha\end{math} de \textit{Krippendorff} & 0.94 & 0.88 & 1.00 & 1.00 & 1.00\\ \hline
\end{tabular}
\caption{\label{tab:presenca-wholqol100}Nível de concordância da presença do vocabulário WHOQOL-100}
\end{table}

\par Validada a presença do vocabulário do questionário \textit{WHOQOL-100} nos comentários do utilizador recolhidos da plataforma do \textit{YouTube}, precedemos de seguida à exibição dos resultados relativamente à concordância da anotação efetuada. 
\par A tabela \ref{tab:anotacoes-wholqol100} mostra os resultados ao nível da concordância entre anotadores nas 23 perguntas selecionadas. Ao analisar estes dados, podemos concluir que existe uma maior concordância dos participantes da área da Medicina, e também que quanto maior é a experiência profissional já vivenciada maior é a concordância nos conceitos anotados. Relativamente às restrinções que se realizaram ao nível da experiência profissional e que estão presentes na tabela \ref{tab:anotacoes-wholqol100}: com mais de 10 anos, estão incluídos três médicos e dois psicólogos; com mais de 15 anos, estão dois médicos e um psicólogo; e com mais de 20 anos, estão dois médicos.

\par Com isto podemos concluir que dada a elevada subjetividade dos conceitos em estudo é preciso ter um trabalho de campo muito acentuado, uma vez que só com a experiência é que estes conceitos vão ficando cada vez mais concretos e definidos na atuação diária destes profissionais de saúde. Para prosseguir para uma anotação manual mais aprofundada, esta deve ser realizada com médicos com mais de 20 anos de experiência profissional, uma vez que obtiveram o \begin{math}\alpha\end{math} de \textit{Krippendorff} com 0.80, o que é indicativo de que o nível de concordância entre eles é muito fiável.

\begin{table}[H]
\centering
\renewcommand{\arraystretch}{1.3}
\begin{tabular}{|m{1.5cm}|m{1.1cm}|m{1.6cm}|m{1.7cm}|m{1.6cm}|m{1.3cm}|m{1.3cm}|m{1.3cm}|}
\hline
\textbf{}&\textbf{Todos}&\textbf{Medicina}&\textbf{Psicologia}&\textbf{Ciências Farmacêuticas}&\textbf{>10 anos exp. profis.} & \textbf{>15 anos exp. profis.} & \textbf{>20 anos exp. profis.}\\ \hline

\begin{math}\alpha\end{math} de \textit{Krippendorff} &
0.46 & 0.58 & 0.42 & 0.38 & 0.47 & 0.58 & 0.80\\ \hline
\end{tabular}
\caption{\label{tab:anotacoes-wholqol100}Nível de concordância na anotação geral efetuada}
\end{table}

\par Relativamente à prevalência de cada um dos conceitos, este estudo foi particularmente útil, uma vez que para além de validar a utilização do vocabulário para fins de mineração da opinião do utilizador, também nos permite antever qual será potencialmente a prevalência de cada um dos conceitos anotados automaticamente pelo sistema. Na tabela \ref{tab:anotacoes-conceitos-wholqol100} apresenta-se as percentagens de deteção de cada um dos conceitos, destacando a participação de todos os intervenientes, e mais especificamente dos sete médicos, e dos dois médicos com mais de 20 anos de carreira que obtiveram uma elevada fiabilidade nos resultados.

\begin{table}[H]
\centering
\renewcommand{\arraystretch}{1.3}
\begin{tabular}{|m{5.5cm}|m{2.0cm}|m{2.0cm}|m{2.5cm}|}
\hline
\textbf{Conceito}&\textbf{Todos}&\textbf{Médicos}&\textbf{> 20 anos experiência profissional}\\ \hline

\textit{Bodily image and Appearance} (Imagem Corporal e Aparência) & 9.50\% & 10.37\% & 8.33\% \\ \hline
\textit{Concentration} (Concentração) & 5.47\% & 3.80\% & 0\%\\ \hline
\textit{Energy} (Energia) & 10.61\% & 10.63\% & 5.21\%\\ \hline
\textit{Fatigue} (Fatiga) & 2.68\% & 2.53\% & 0\%\\ \hline
\textit{Learning} (Aprendizagem) & 6.81\% & 6.84\% & 5.21\% \\ \hline
\textit{Memory} (Memória) & 5.47\% & 5.06\% & 4.17\% \\ \hline
\textit{Negative feelings} (Percepções negativas) & 6.36\% & 7.34\% & 11.46\% \\ \hline
\textit{Pain and discomfort} (Dor e desconforto) & 3.12\% & 1.01\% & 2.08\%  \\ \hline
\textit{Personal relationships} (Relações pessoais)& 8.82\% & 10.38\% & 14.58\% \\ \hline
\textit{Positive feelings} (Percepções positivas)& 13.74\% & 17.21\% & 22.91\% \\ \hline
\textit{Self-esteem} (Auto-estima) & 12.84\% & 13.67\%& 15.625\% \\ \hline
\textit{Sexual activity} (Atividade sexual)& 0.44\% & 0\% & 0\%\\ \hline
\textit{Sleep and rest} (Dormir e Descansar)& 2.57\% & 3.04\%& 4.17\% \\ \hline
\textit{Social Support} (Apoio Social) & 7.60\% & 5.57\%& 3.125\% \\ \hline
\textit{Thinking} (Pensamento) & 3.91\% & 2.53\% & 3.125\% \\ \hline

\end{tabular}
\caption{\label{tab:anotacoes-conceitos-wholqol100}Anotação manual dos conceitos do WHOQOL-100}
\end{table}
